\counterwithin*{equation}{section}
\counterwithin*{equation}{subsection}
\counterwithin*{equation}{subsubsection}

\appendix
\section{Appendix} \label{appendix}

\subsection{Full derivation of a hypothetical unique identity cubrix in $M_2 (\mathbb{K})$.} \label{appendix-1}

The expression $A = IIA$ can be expanded using the definition of the cubrix product:

\begin{equation}
A_{111} = I_{111} I_{111} A_{111} + I_{121} I_{211} A_{112}
\end{equation}

\begin{equation}
A_{112} = I_{112} I_{112} A_{111} + I_{122} I_{212} A_{112}
\end{equation}

\begin{equation}
A_{121} = I_{111} I_{121} A_{121} + I_{121} I_{221} A_{122}
\end{equation}

\begin{equation}
A_{122} = I_{112} I_{122} A_{121} + I_{122} I_{222} A_{122}
\end{equation}

\begin{equation}
A_{211} = I_{211} I_{111} A_{211} + I_{221} I_{211} A_{212}
\end{equation}

\begin{equation}
A_{212} = I_{212} I_{112} A_{211} + I_{222} I_{212} A_{212}
\end{equation}

\begin{equation}
A_{221} = I_{211} I_{121} A_{221} + I_{221} I_{221} A_{222}
\end{equation}

\begin{equation}
A_{222} = I_{212} I_{122} A_{221} + I_{222} I_{222} A_{222}
\end{equation}

Using this system, we get the following:

\begin{enumerate}[(a)]
	\item By $(1) \Rightarrow I_{111} = 1$.
	\item By $(8) \Rightarrow I_{222} = 1$.
	\item By $(3)$ and $(a) \Rightarrow I_{121} = 1$.
	\item By $(3)$ and $(c) \Rightarrow I_{221} = 0$.
	\item By $(6)$ and $(b) \Rightarrow I_{212} = 1$.
	\item By $(2)$ and $(e) \Rightarrow I_{122} = 1$.
\end{enumerate}

By $(8)$, either $I_{212}$ or $I_{122}$ must be equal to $0$, but by $(e)$ and $(f)$, $I_{212} = I_{122} = 1 \neq 0$. The same contradiction occurs when trying with $IAI$ and $AII$.

\subsection{Full derivation of the product by an identity pair in $M_2 (\mathbb{K})$.} \label{appendix-2}

\begin{equation}
A_{111} = A_{111} I_{111} J_{111} + A_{121} I_{211} J_{112}
\end{equation}

\begin{equation}
A_{112} = A_{112} I_{112} J_{111} + A_{122} I_{212} J_{112}
\end{equation}

\begin{equation}
A_{121} = A_{111} I_{121} J_{121} + A_{121} I_{221} J_{122}
\end{equation}

\begin{equation}
A_{122} = A_{112} I_{122} J_{121} + A_{122} I_{222} J_{122}
\end{equation}

\begin{equation}
A_{211} = A_{211} I_{111} J_{211} + A_{221} I_{211} J_{212}
\end{equation}

\begin{equation}
A_{212} = A_{212} I_{112} J_{211} + A_{222} I_{212} J_{212}
\end{equation}

\begin{equation}
A_{221} = A_{211} I_{121} J_{221} + A_{221} I_{221} J_{222}
\end{equation}

\begin{equation}
A_{222} = A_{212} I_{122} J_{221} + A_{222} I_{222} J_{222}
\end{equation}

In an analogous way, we get that:

\begin{enumerate}[(a)]
	\item By $(1) \Rightarrow I_{111} \neq 0 \neq J_{111}$ y $I_{111} = J_{111}^{-1}$.
	\item By $(8) \Rightarrow I_{222} \neq 0 \neq J_{222}$ y $I_{222} = J_{222}^{-1}$.
	\item By $(2)$ and $(a) \Rightarrow I_{112} = J_{111}^{-1} = I_{111} \neq 0$.
	\item By $(6)$ and $(c) \Rightarrow I_{112} = J_{211}^{-1} \neq 0 \Rightarrow J_{211} \neq 0$.
	\item By $(5)$ and $(d) \Rightarrow J_{211} = I_{111}^{-1} = J_{111}$.
	\item By $(7)$ and $(b) \Rightarrow I_{221} = J_{222}^{-1} \neq 0$.
	\item By $(3)$, $(4)$ and $(f) \Rightarrow J_{122} = I_{221}^{-1} = I_{222}^{-1} = J_{222}^{-1}$.
	\item By $(g)$ and $(b) \Rightarrow I_{222} = J_{222}^{-1} = I_{222}^{-1} \Rightarrow I_{222}^2 = 1 \Rightarrow I_{222} = 1 = J_{222}$.
\end{enumerate}

\newpage

To sum up:

\begin{itemize}
	\item $I_{221} = J_{122} = I_{222} = J_{222} = 1$.
	\item $J_{211}^{-1} = I_{111} = J_{111}^{-1} = I_{112}$.
	\item The rest of the terms:

	\begin{itemize}
		\item $I_{211} J_{112} = 0$.
		\item $I_{212} J_{112} = 0$.
		\item $I_{121} J_{121} = 0$.
		\item $I_{122} J_{121} = 0$.
		\item $I_{211} J_{212} = 0$.
		\item $I_{212} J_{212} = 0$.
		\item $I_{121} J_{221} = 0$.
		\item $I_{122} J_{221} = 0$.
	\end{itemize}
\end{itemize}

To standardize, we can assume that the elements that don't cancel with others will be equal to $1$, whereas the ones which do, will be equal to $0$.

\begin{itemize}
	\item $1 = I_{221} = J_{122} = I_{222} = J_{222} = J_{211}^{-1} = I_{111} = J_{111}^{-1} = I_{112}$.
	\item $0 = I_{211} = J_{112} = I_{212} = I_{121} = J_{121} = I_{122} = J_{212} = J_{221}$.
\end{itemize}

We see that the following is always satisfied:

\begin{itemize}
	\item If $j = 1 \rightarrow I_{1jk} = J_{ij1} = 1$ and $I_{2jk} = J_{ij2} = 0$.
	\item If $j = 2 \rightarrow I_{1jk} = J_{ij1} = 0$ and $I_{2jk} = J_{ij2} = 1$.
\end{itemize}

The pattern becomes trivial if we see what happens with a $3 \times 3 \times 3$ cubrix:

$$(AIJ)_{ijk} = A_{i1k} I_{1jk} J_{ij1} + A_{i2k} I_{2jk} J_{ij2} + A_{i3k} I_{3jk} J_{ij3}$$

\begin{itemize}
	\item If $j = 1 \rightarrow I_{1jk} = J_{ij1} = 1$ and $I_{2jk} = J_{ij2} = 0$ and $I_{3jk} = J_{ij3} = 0$.
	\item If $j = 2 \rightarrow I_{1jk} = J_{ij1} = 0$ and $I_{2jk} = J_{ij2} = 1$ and $I_{3jk} = J_{ij3} = 0$.
	\item If $j = 3 \rightarrow I_{1jk} = J_{ij1} = 0$ and $I_{2jk} = J_{ij2} = 0$ and $I_{3jk} = J_{ij3} = 1$.
\end{itemize}

That is, for $(AIJ)_{ijk} = A_{ijk}$ to be true, then $I_{ljk} = J_{ijl} = \delta_{jl}$ must also be true.

As we discovered in section \ref{identity-pair}, we can replace the subindices and arrive to the three kroneckers.

\subsection{Full derivation of inverse pairs.} \label{appendix-3}

Given $A$, our objective is to find its inverse pairs over two components in any arrangement of factors. We will treat products which commute $I$ and $J$ as equal cases (justifying $AIJ$ will also justify $AJI$), as this will only commute the names we give to the values.

\subsubsection{$AIJ$ over $i, j$.}

By definition:

$$(AIJ)_{ijk} = \delta_{ij} = \sum\limits^{n}_{l = 1} A_{ilk} I_{ljk} J_{ijl}$$

We can explain this expression using words: ``If $i \neq j$, then $I_{ljk} J_{ijl} = 0$. If $i = j$, and if $l$ takes the value of $j$, then $I_{ljk} = A^{-1}_{ilk}$ and $J_{ijl} = 1$''. It's vital noting that the fact that $l$ takes the value of $j$ implies that the first subindex of $I_{ljk}$ also takes the value of $j$. This is equivalent to $i$ being equal to $j$ (but \textbf{only} in $I$'s case, as in other terms the iterator $l$ will occupy another subindex's position). With this in mind, expressing the same using kronecker deltas is automatic:

$$I_{ijk} = \delta_{ij} \delta_{ij} A_{ijk}^{-1} = \delta_{ij} A_{ijk}^{-1}$$
$$J_{ijk} = \delta_{ij} \approx \delta_{ij} \delta_{jk}$$

In $I$, the first $\delta_{ij}$ represents the condition imposed by the kronecker delta we aspire to get, whereas the second $\delta_{ij}$ represents the condition imposed by the iterator $l$, but expressed in terms of $I$'s subindices. Obviously, $\delta_{ij} \delta_{ij} = \delta_{ij}$.

In $J$, the term $\delta_{jk}$ represents the condition that ``...if $i = j$ but $l$ doesn't take the value of $j$...'' (which in $J$ is equivalent to saying that $k$ doesn't take the value of $j$). This term is unnecesary, as it's only required that the product gets cancelled (and not necessarily both factors), and this is already satisfied thanks to the $\delta_{ij}$ present in $I$. The imposed conditions over $I$'s and $J$'s values are perfectly represented in the following algorithm:

\begin{lstlisting}[escapeinside={(*}{*)}]
If i == j:
  If l == j:
    (*$I_{ljk} = A_{ilk}^{-1}$*)
    (*$J_{ijl} = 1$*)
  Else:
    (*$I_{ljk} J_{ijl} = 0$*)
Else:
  (*$I_{ljk} J_{ijl} = 0$*)	
\end{lstlisting}

In this article we omit the inclusion of the second factor, both for aesthetical and for practical reasons (see \ref{appendix-4}).

\subsubsection{$AIJ$ over $j, k$.}

The process is almost identical to the previous one, but commuting $I$ and $J$ (the inverse values will be contained in $J$, and the neutral ones in $I$). The reason to do this lies in the need to coordinate $I$'s and $J$'s subindices with the iterator such that we get easily malleable expressions like $\delta_{jk} \delta_{jk}$

\subsubsection{$AIJ$ over $i, k$.}

This case requires some attention. Let $a = i = k$ (if $i \neq k$, the product $I_{ijk} J_{ijk}$ will equal zero):

$$\delta_{aa} = 1 = \sum\limits_{l = 1}^{n} A_{ala} I_{lja} J_{ajl}$$

Until now, $I_{ijk}$ had always been dependant on the inverse of $A_{ijk}$, however, this case breaks that relationship. Because of that, we must find other criterion for $I$ and $J$ not to cancel. Once again, by a mixture of aesthetical reasons and process of elimination, we establish the following: when $I$'s and $J$'s indices are equal, their product with $A_{ijk}$ will yield $1$ (and when not, $0$).

It's easy to find out that $I$'s and $J$'s subindices will be equal when $l$ takes the value of $a$. So:

$$I_{aja} = A_{aaa}^{-1}$$
$$J_{aja} = 1$$

Using $i$, $j$ and $k$:

$$I_{ijk} = \delta_{ik} A_{iii}^{-1} = \delta_{ik} A_{kkk}^{-1}$$
$$J_{ijk} = \delta_{ik}$$

\subsubsection{Et cetera.}

We have been working with $AIJ$, but we could have done the same procedures over $IAJ$ and $IJA$. Every case can be solved in an analogous way to one of the three previous methods, and the final result is Table \ref{tabla-explicita} on section \ref{inverse-all-pairs}.

\subsection{Simplification of inverse pairs.} \label{appendix-4}

The expressions for inverse pairs can be simplified quite remarkably by just reordering some terms.

As our first and only example, we'll use the case of $IJA$ over $i, k$. According to Table \ref{table-explicita}:

$$I_{ijk} = \delta_{ik}$$
$$J_{ijk} = \delta_{ik} A_{ijk}^{-1}$$

Plugging that into the product's definition:

$$(IJA)_{ijk} = (\delta_{ik})_{ilk} \cdot (\delta_{ik} A_{ijk}^{-1})_{ljk} \cdot A$$

And plugging in the subindices:

$$(IJA)_{ijk} = (\delta_{ik}) \cdot (\delta_{lk} A_{ljk}^{-1}) \cdot A$$

Because $\mathbb{K}$ is associative and commutative, we can rewrite this expression such that:

$$(IJA)_{ijk} = (\delta_{ik} \delta_{lk}) \cdot A_{ljk}^{-1} \cdot A$$

And this is equivalent to having another inverse pair $(I', J')$ where:

$$I'_{ijk} = \delta_{ik} \delta_{jk}$$
$$J'_{ijk} = (A_{ijk})^{-1}$$

($l$ will be plugged into $j$ in $I$ and into $i$ in $J$).

To provide some semantic meaning to this inverse pair, we can rename $I'$ to be $\Delta$ and $J'$ to $A^{-1}$.

The same procedure is applicable to the other cases (excluding the ones on Table \ref{tabla-explicita}'s positive diagonal), producing Table \ref{tabla-simplificada} from section \ref{inverse-all-pairs}.
