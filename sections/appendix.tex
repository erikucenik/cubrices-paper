\counterwithin*{equation}{section}
\counterwithin*{equation}{subsection}
\counterwithin*{equation}{subsubsection}

\appendix
\section{Apéndice} \label{appendix}

\subsection{Desarrollo completo del producto por una hipotética cubriz identidad única en $M_2 (\mathbb{K})$.} \label{appendix-1}

La expresión $A = IIA$ puede ser desarrollada por la definición del producto cubricial:

\begin{equation}
A_{111} = I_{111} I_{111} A_{111} + I_{121} I_{211} A_{112}
\end{equation}

\begin{equation}
A_{112} = I_{112} I_{112} A_{111} + I_{122} I_{212} A_{112}
\end{equation}

\begin{equation}
A_{121} = I_{111} I_{121} A_{121} + I_{121} I_{221} A_{122}
\end{equation}

\begin{equation}
A_{122} = I_{112} I_{122} A_{121} + I_{122} I_{222} A_{122}
\end{equation}

\begin{equation}
A_{211} = I_{211} I_{111} A_{211} + I_{221} I_{211} A_{212}
\end{equation}

\begin{equation}
A_{212} = I_{212} I_{112} A_{211} + I_{222} I_{212} A_{212}
\end{equation}

\begin{equation}
A_{221} = I_{211} I_{121} A_{221} + I_{221} I_{221} A_{222}
\end{equation}

\begin{equation}
A_{222} = I_{212} I_{122} A_{221} + I_{222} I_{222} A_{222}
\end{equation}

En base a ese sistema derivamos lo siguiente:

\begin{enumerate}[(a)]
	\item Por $(1) \Rightarrow I_{111} = 1$.
	\item Por $(8) \Rightarrow I_{222} = 1$.
	\item Por $(3)$ y $(a) \Rightarrow I_{121} = 1$.
	\item Por $(3)$ y $(c) \Rightarrow I_{221} = 0$.
	\item Por $(6)$ y $(b) \Rightarrow I_{212} = 1$.
	\item Por $(2)$ y $(e) \Rightarrow I_{122} = 1$.
\end{enumerate}

Por $(8)$, o $I_{212}$ o $I_{122}$ deben ser $0$, pero por $(e)$ y $(f)$, $I_{212} = I_{122} = 1 \neq 0$. Surge la misma contradicción al probar con $IAI$ y $AII$.

\subsection{Desarrollo completo del producto por un par identidad en $M_2 (\mathbb{K})$.} \label{appendix-2}

\begin{equation}
A_{111} = A_{111} I_{111} J_{111} + A_{121} I_{211} J_{112}
\end{equation}

\begin{equation}
A_{112} = A_{112} I_{112} J_{111} + A_{122} I_{212} J_{112}
\end{equation}

\begin{equation}
A_{121} = A_{111} I_{121} J_{121} + A_{121} I_{221} J_{122}
\end{equation}

\begin{equation}
A_{122} = A_{112} I_{122} J_{121} + A_{122} I_{222} J_{122}
\end{equation}

\begin{equation}
A_{211} = A_{211} I_{111} J_{211} + A_{221} I_{211} J_{212}
\end{equation}

\begin{equation}
A_{212} = A_{212} I_{112} J_{211} + A_{222} I_{212} J_{212}
\end{equation}

\begin{equation}
A_{221} = A_{211} I_{121} J_{221} + A_{221} I_{221} J_{222}
\end{equation}

\begin{equation}
A_{222} = A_{212} I_{122} J_{221} + A_{222} I_{222} J_{222}
\end{equation}

De manera análoga, derivamos que:

\begin{enumerate}[(a)]
	\item Por $(1) \Rightarrow I_{111} \neq 0 \neq J_{111}$ y $I_{111} = J_{111}^{-1}$.
	\item Por $(8) \Rightarrow I_{222} \neq 0 \neq J_{222}$ y $I_{222} = J_{222}^{-1}$.
	\item Por $(2)$ y $(a) \Rightarrow I_{112} = J_{111}^{-1} = I_{111} \neq 0$.
	\item Por $(6)$ y $(c) \Rightarrow I_{112} = J_{211}^{-1} \neq 0 \Rightarrow J_{211} \neq 0$.
	\item Por $(5)$ y $(d) \Rightarrow J_{211} = I_{111}^{-1} = J_{111}$.
	\item Por $(7)$ y $(b) \Rightarrow I_{221} = J_{222}^{-1} \neq 0$.
	\item Por $(3)$, $(4)$ y $(f) \Rightarrow J_{122} = I_{221}^{-1} = I_{222}^{-1} = J_{222}^{-1}$.
	\item Por $(g)$ y $(b) \Rightarrow I_{222} = J_{222}^{-1} = I_{222}^{-1} \Rightarrow I_{222}^2 = 1 \Rightarrow I_{222} = 1 = J_{222}$.
\end{enumerate}

\newpage

Recapitulando:

\begin{itemize}
	\item $I_{221} = J_{122} = I_{222} = J_{222} = 1$.
	\item $J_{211}^{-1} = I_{111} = J_{111}^{-1} = I_{112}$.
	\item El resto de términos:

	\begin{itemize}
		\item $I_{211} J_{112} = 0$.
		\item $I_{212} J_{112} = 0$.
		\item $I_{121} J_{121} = 0$.
		\item $I_{122} J_{121} = 0$.
		\item $I_{211} J_{212} = 0$.
		\item $I_{212} J_{212} = 0$.
		\item $I_{121} J_{221} = 0$.
		\item $I_{122} J_{221} = 0$.
	\end{itemize}
\end{itemize}

Por cuestiones de estandarización, podemos asumir que los elementos que no se anulen con otros serán iguales a $1$, mientras que aquéllos que sí se anulen, serán $0$.

\begin{itemize}
	\item $1 = I_{221} = J_{122} = I_{222} = J_{222} = J_{211}^{-1} = I_{111} = J_{111}^{-1} = I_{112}$.
	\item $0 = I_{211} = J_{112} = I_{212} = I_{121} = J_{121} = I_{122} = J_{212} = J_{221}$.
\end{itemize}

Vemos que siempre se cumple que:

\begin{itemize}
	\item Si $j = 1 \rightarrow I_{1jk} = J_{ij1} = 1$ y $I_{2jk} = J_{ij2} = 0$.
	\item Si $j = 2 \rightarrow I_{1jk} = J_{ij1} = 0$ y $I_{2jk} = J_{ij2} = 1$.
\end{itemize}

El patrón se vuelve trivial si vemos qué ocurre con una cubriz $3 \times 3 \times 3$:

$$(AIJ)_{ijk} = A_{i1k} I_{1jk} J_{ij1} + A_{i2k} I_{2jk} J_{ij2} + A_{i3k} I_{3jk} J_{ij3}$$

\begin{itemize}
	\item Si $j = 1 \rightarrow I_{1jk} = J_{ij1} = 1$ y $I_{2jk} = J_{ij2} = 0$ y $I_{3jk} = J_{ij3} = 0$.
	\item Si $j = 2 \rightarrow I_{1jk} = J_{ij1} = 0$ y $I_{2jk} = J_{ij2} = 1$ y $I_{3jk} = J_{ij3} = 0$.
	\item Si $j = 3 \rightarrow I_{1jk} = J_{ij1} = 0$ y $I_{2jk} = J_{ij2} = 0$ y $I_{3jk} = J_{ij3} = 1$.
\end{itemize}

Es decir, para que $(AIJ)_{ijk} = A_{ijk}$, debe cumplirse que $I_{ljk} = J_{ijl} = \delta_{jl}$.

Como descubrimos en la sección \ref{identity-pair}, gracias a que los subíndices son mudos podemos llegar a las tres Kronecker.

\subsection{Desarrollo completo de los pares inversa.} \label{appendix-3}

Dado $A$, nuestro objetivo es hallar sus pares inversa sobre dos componentes en cualquier orden de producto. Trataremos los productos que conmuten $I$ con $J$ como casos iguales (justificar $AIJ$ justificará también $AJI$), dado que el resultado de esto será solo conmutar los nombres de los valores.

\subsubsection{$AIJ$ sobre $i, j$.}

Por definición:

$$(AIJ)_{ijk} = \delta_{ij} = \sum\limits^{n}_{l = 1} A_{ilk} I_{ljk} J_{ijl}$$

Podemos explicar esta expresión en palabras: ``Si $i \neq j$, entonces $I_{ljk} J_{ijl} = 0$. Si $i = j$, y si además $l$ toma el valor de $j$, entonces $I_{ljk} = A^{-1}_{ilk}$ y $J_{ijl} = 1$''. Es vital notar que el hecho de que $l$ tome el valor de $j$ implica que el primer subíndice de $I_{ljk}$ también toma el valor de $j$. Al ser los subíndices mudos, esto equivale a que $i$ sea igual a $j$ (pero \textbf{solo} en el caso de $I$, pues en otros términos el iterador $l$ ocupará la posición de otro subíndice). Con esto en mente, expresar lo mismo utilizando deltas de kronecker es automático:

$$I_{ijk} = \delta_{ij} \delta_{ij} A_{ijk}^{-1} = \delta_{ij} A_{ijk}^{-1}$$
$$J_{ijk} = \delta_{ij} \approx \delta_{ij} \delta_{jk}$$

En $I$, el primer $\delta_{ij}$ representa la condición que impone la delta de kronecker a la que aspiramos a llegar, mientras que el segundo $\delta_{ij}$ representa la condición que impone el iterador $l$ pero expresada en términos de subíndices de $I$. Como es evidente, $\delta_{ij} \delta_{ij} = \delta_{ij}$.

En $J$, el término $\delta_{jk}$ representa la condición de ``...si $i = j$ pero $l$ no toma el valor de $j$...'' (lo que en $J$ equivale a que $k$ no tome el valor de $j$). Este término es innecesario, ya que solo se requiere la anulación del producto (y no de ambos factores), y esto ya se satisface gracias al $\delta_{ij}$ presente en $I$. Las condiciones impuestas sobre los valores de $I$ y $J$ quedan perfectamente representadas en el siguiente algoritmo:

\begin{lstlisting}[escapeinside={(*}{*)}]
Si i == j:
  Si l == j:
    (*$I_{ljk} = A_{ilk}^{-1}$*)
    (*$J_{ijl} = 1$*)
  Si no:
    (*$I_{ljk} J_{ijl} = 0$*)
Si no:
  (*$I_{ljk} J_{ijl} = 0$*)	
\end{lstlisting}

En este artículo omitimos la inclusión del segundo factor, tanto por motivos estéticos como prácticos (véase la siguiente sección del anexo).

\subsubsection{$AIJ$ sobre $j, k$.}

El proceso es casi idéntico al anterior, pero conmutaremos $I$ con $J$ (los valores inversos pasarán a estar en $J$, y los neutros en $I$). La necesidad de hacer esto yace en el deseo de coordinar los subíndices de $I$ y $J$ con el iterador de tal manera que obtengamos expresiones fácilmente maleables como $\delta_{jk} \delta_{jk}$.

\subsubsection{$AIJ$ sobre $i, k$.}

Este caso requiere algo de atención. Sea $a = i = k$ (en caso de que $i \neq k$, el producto de $I_{ijk}$ y $J_{ijk}$ será cero):

$$\delta_{aa} = 1 = \sum\limits_{l = 1}^{n} A_{ala} I_{lja} J_{ajl}$$

Hasta ahora, $I_{ijk}$ siempre había dependido del inverso de $A_{ijk}$, sin embargo este caso rompe esa relación. Es por eso que tenemos que buscar otro criterio para no anular a $I$ y a $J$. Nuevamente por una mezcla de motivos estéticos y proceso de eliminación, establecemos lo siguiente: que cuando los índices de $I$ y $J$ sean iguales, su producto con $A_{ijk}$ dará $1$ (y cuando no, $0$).

Es fácil descubrir que los subíndices de $I$ y $J$ serán iguales cuando $l$ tome el valor de $a$. Es decir:

$$I_{aja} = A_{aaa}^{-1}$$
$$J_{aja} = 1$$

O expresado con $i$, $j$ y $k$:

$$I_{ijk} = \delta_{ik} A_{iii}^{-1} = \delta_{ik} A_{kkk}^{-1}$$
$$J_{ijk} = \delta_{ik}$$

\subsubsection{Et cetera.}

Hemos trabajado con $AIJ$, pero podríamos haber hecho los mismos procedimientos sobre $IAJ$ e $IJA$. Todos los casos se pueden resolver de forma análoga con uno de los tres métodos anteriores, y el resultado final es la primera tabla de la sección \ref{inverse-all-pairs}.

\subsection{Simplificación de los pares inversa.} \label{appendix-4}

Las expresiones de los pares inversa pueden ser simplificadas con bastante éxito tan solo reordenando algunos términos.

Como primer y único ejemplo usaremos el caso de $IJA$ sobre $i, k$. Según la tabla:

$$I_{ijk} = \delta_{ik}$$
$$J_{ijk} = \delta_{ik} A_{ijk}^{-1}$$

Insertando eso en la definición del producto:

$$(IJA)_{ijk} = (\delta_{ik})_{ilk} \cdot (\delta_{ik} A_{ijk}^{-1})_{ljk} \cdot A$$

Y sustituyendo los subíndices:

$$(IJA)_{ijk} = (\delta_{ik}) \cdot (\delta_{lk} A_{ljk}^{-1}) \cdot A$$

Gracias a que $\mathbb{K}$ es asociativo y conmutativo, podemos reescribir esta expresión tal que:

$$(IJA)_{ijk} = (\delta_{ik} \delta_{lk}) \cdot A_{ljk}^{-1} \cdot A$$

Y esto es equivalente a tener otro par inversa $(I', J')$ donde:

$$I'_{ijk} = \delta_{ik} \delta_{jk}$$
$$J'_{ijk} = (A_{ijk})^{-1}$$

($l$ se sustituiría en $j$ para $I$ y en $i$ para $J$).

Para dar algo de sentido semántico a este par inversa, podemos llamar $\Delta$ a $I'$ y $A^{-1}$ a $J'$.

La misma receta es aplicable al resto de casos (excluyendo aquéllos de la diagonal positiva en la primera tabla de la sección \ref{inverse-all-pairs}), y produce la segunda tabla de la sección \ref{inverse-all-pairs}.
