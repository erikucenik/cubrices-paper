\section{Notion of inverse.} \label{inverse}

\subsection{Inverse pair.} \label{inverse-pair}

As we did before, we'll find inspiration in the field of matrices, where the inverse matrix $A^{-1}$ of a nother matrix $A$ is that which satisfies that ${A\cdot A^{-1} = A^{-1} \cdot A = I}$. The multiplicity of identity cubrices has already become evident, so we should expect the same for inverse cubrices.

We say that $I$ and $J$ form an \textit{inverse pair} in $AIJ$ over $i, j$ if $(AIJ) = \Delta_{ij}$. We can say the same for pairs in other orders ($IAJ$, $IJA$) and over other kronecker cubrices ($\Delta_{jk}$, $\Delta_{ik}$). Note that commutativity isn't insured.

\subsection{All inverse pairs.} \label{inverse-all-pairs}

After a somewhat repetitive derivation (see Appendix \ref{appendix-3}), we can gather all non-redundant inverse pairs in the following table.

\bgroup
\begin{table}[H] 
\centering
\captionsetup{labelformat=empty}
\caption{(Table 1) Inverse pairs.}
\label{tabla-explicita}
\def\arraystretch{1.5}%  1 is the default, change whatever you need
\begin{tabular}{|c|c|c|c|} 
	\hline
	& $\Delta_{ij}$ & $\Delta_{jk}$ & $\Delta_{ik}$ \\
	\hline

	$AIJ$
	& \begin{tabular}{c} $I_{ijk} = \delta_{ij} A_{ijk}^{-1}$ \\ $J_{ijk} = \delta_{ij}$ \end{tabular}
	& \begin{tabular}{c} $I_{ijk} = \delta_{jk}$ \\ $J_{ijk} = \delta_{jk} A_{ijk}^{-1}$ \end{tabular}
	& \begin{tabular}{c} $I_{ijk} = \delta_{ik} A_{kkk}^{-1}$ \\ $J_{ijk} = \delta_{ik}$ \end{tabular} \\
	\hline

	$IAJ$
	& \begin{tabular}{c} $I_{ijk} = \delta_{ij} A_{ijk}^{-1}$ \\ $J_{ijk} = \delta_{ij}$ \end{tabular}
	& \begin{tabular}{c} $I_{ijk} = \delta_{jk} A_{jjj}^{-1}$ \\ $J_{ijk} = \delta_{jk}$ \end{tabular}
	& \begin{tabular}{c} $I_{ijk} = \delta_{ik}$ \\ $J_{ijk} = \delta_{ik} A_{ijk}^{-1}$ \end{tabular} \\
	\hline

	$IJA$
	& \begin{tabular}{c} $I_{ijk} = \delta_{ij} A_{iii}^{-1}$ \\ $J_{ijk} = \delta_{ij}$ \end{tabular}
	& \begin{tabular}{c} $I_{ijk} = \delta_{jk} A_{ijk}^{-1}$ \\ $J_{ijk} = \delta_{jk}$ \end{tabular}
	& \begin{tabular}{c} $I_{ijk} = \delta_{ik}$ \\ $J_{ijk} = \delta_{ik} A_{ijk}^{-1}$ \end{tabular} \\
	\hline
\end{tabular}
\end{table}
\egroup

It's easy to note that with some simple renamings we can obtain a much cleaner result (see Appendix \ref{appendix-4}). Let $(A^{-1})_{ijk} = (A_{ijk})^{-1}$ and $\Delta = (\delta_{ij}\delta_{jk})$. We can distribute their products with $A$ in different columns according to the kronecker that comes out of each one.

\bgroup
\begin{table}[H]
\centering
\captionsetup{labelformat=empty}
\caption{(Table 2) Simplified inverse pairs.}
\label{tabla-simplificada}
\def\arraystretch{1.5}%  1 is the default, change whatever you need
\begin{tabular}{|c|c|c|c|} 
	\hline
	$\Delta_{ij}$ & $\Delta_{jk}$ & $\Delta_{ik}$ \\
	\hline
	\begin{tabular}{c} $(A \cdot A^{-1} \cdot \Delta)$ \\ $(A^{-1} \cdot A \cdot \Delta)$ \end{tabular} &
	\begin{tabular}{c} $(A \cdot \Delta \cdot A^{-1})$ \\ $(A^{-1} \cdot \Delta \cdot A)$ \end{tabular} &
	\begin{tabular}{c} $(\Delta \cdot A \cdot A^{-1})$ \\ $(\Delta \cdot A^{-1} \cdot A)$ \end{tabular} \\
	\hline
\end{tabular}
\end{table}
\egroup

Note the exclusion of the first table's positive diagonal. Those expressions can't be transformed in such clean terms. We could say that $((\delta_{ab} A^{-1}_{aaa}, (\delta_{ab}))$ forms an inverse pair for those arrangements, but we consider beauty to be more important than completeness.

Analyzing the last table, it seems evident that commuting $A$ with $A^{-1}$ doesn't alter the result. In other words: it's $\Delta$'s position the one that determines which kronecker cubrix the product will yield. For example, $\Delta_{jk}$ is produced when $\Delta$ is in the second position. This holds some intuitive sense, as in this position, its subindex $i$ gets replaced by the iterator $l$ from the sum, leaving $j$ and $k$ as the only components.
