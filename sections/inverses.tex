\section{Noción de inversa.} \label{inverse}

\subsection{Par inversa.} \label{inverse-pair}

Al igual que antes, buscamos inspiración en el área de las matrices, donde la matriz inversa $A^{-1}$ a otra matriz $A$ es aquélla que satisface que ${A\cdot A^{-1} = A^{-1} \cdot A = I}$. Ya ha quedado en evidencia la multiplicidad de las cubrices identidad, así que hemos de esperar lo mismo para las cubrices inversas.

Decimos que $I$ y $J$ forman un \textit{par inversa} en $AIJ$ sobre $i, j$ si $(AIJ) = \Delta_{ij}$. Podemos afirmar lo mismo para pares en otros órdenes del producto ($IAJ$, $IJA$) y sobre otras cubrices kronecker ($\Delta_{jk}$, $\Delta_{ik}$). Nótese que la conmutatividad no está asegurada.

\subsection{Todos los pares inversas.} \label{inverse-all-pairs}

Tras un desarrollo algo repetitivo (ver Apéndice \ref{appendix-3}), recogeremos en la siguiente tabla todos los pares inversa no redundantes existentes.

\bgroup
\begin{table}[H] 
\centering
\captionsetup{labelformat=empty}
\caption{(Tabla 1) Pares inversa.}
\label{tabla-explicita}
\def\arraystretch{1.5}%  1 is the default, change whatever you need
\begin{tabular}{|c|c|c|c|} 
	\hline
	& $\Delta_{ij}$ & $\Delta_{jk}$ & $\Delta_{ik}$ \\
	\hline

	$AIJ$
	& \begin{tabular}{c} $I_{ijk} = \delta_{ij} A_{ijk}^{-1}$ \\ $J_{ijk} = \delta_{ij}$ \end{tabular}
	& \begin{tabular}{c} $I_{ijk} = \delta_{jk}$ \\ $J_{ijk} = \delta_{jk} A_{ijk}^{-1}$ \end{tabular}
	& \begin{tabular}{c} $I_{ijk} = \delta_{ik} A_{kkk}^{-1}$ \\ $J_{ijk} = \delta_{ik}$ \end{tabular} \\
	\hline

	$IAJ$
	& \begin{tabular}{c} $I_{ijk} = \delta_{ij} A_{ijk}^{-1}$ \\ $J_{ijk} = \delta_{ij}$ \end{tabular}
	& \begin{tabular}{c} $I_{ijk} = \delta_{jk} A_{jjj}^{-1}$ \\ $J_{ijk} = \delta_{jk}$ \end{tabular}
	& \begin{tabular}{c} $I_{ijk} = \delta_{ik}$ \\ $J_{ijk} = \delta_{ik} A_{ijk}^{-1}$ \end{tabular} \\
	\hline

	$IJA$
	& \begin{tabular}{c} $I_{ijk} = \delta_{ij} A_{iii}^{-1}$ \\ $J_{ijk} = \delta_{ij}$ \end{tabular}
	& \begin{tabular}{c} $I_{ijk} = \delta_{jk} A_{ijk}^{-1}$ \\ $J_{ijk} = \delta_{jk}$ \end{tabular}
	& \begin{tabular}{c} $I_{ijk} = \delta_{ik}$ \\ $J_{ijk} = \delta_{ik} A_{ijk}^{-1}$ \end{tabular} \\
	\hline
\end{tabular}
\end{table}
\egroup

Es fácil notar que con unos simples cambios de nombre podemos obtener un resultado mucho más limpio (ver Apéndice \ref{appendix-4}). Sea $(A^{-1})_{ijk} = (A_{ijk})^{-1}$ y $\Delta = (\delta_{ij}\delta_{jk})$. Podemos repartir sus productos con $A$ en distintas columnas según la kronecker que resulte del mismo.

\bgroup
\begin{table}[H]
\centering
\captionsetup{labelformat=empty}
\caption{(Tabla 2) Pares inversa simplificados.}
\label{tabla-simplificada}
\def\arraystretch{1.5}%  1 is the default, change whatever you need
\begin{tabular}{|c|c|c|c|} 
	\hline
	$\Delta_{ij}$ & $\Delta_{jk}$ & $\Delta_{ik}$ \\
	\hline
	\begin{tabular}{c} $(A \cdot A^{-1} \cdot \Delta)$ \\ $(A^{-1} \cdot A \cdot \Delta)$ \end{tabular} &
	\begin{tabular}{c} $(A \cdot \Delta \cdot A^{-1})$ \\ $(A^{-1} \cdot \Delta \cdot A)$ \end{tabular} &
	\begin{tabular}{c} $(\Delta \cdot A \cdot A^{-1})$ \\ $(\Delta \cdot A^{-1} \cdot A)$ \end{tabular} \\
	\hline
\end{tabular}
\end{table}
\egroup

Nótese la exclusión de la diagonal positiva de la primera tabla presentada. Aquéllas expresiones no pueden ser transformadas en términos tan limpios. Si bien podríamos afirmar que $((\delta_{ab} A^{-1}_{aaa}, (\delta_{ab}))$ constituye un par inversa para esos ordenes, consideramos que la belleza ha de primar sobre la completitud.

Analizando la última tabla, parece evidente que conmutar $A$ con $A^{-1}$ no altera el resultado. En otras palabras: es la posición de $\Delta$ la que determina qué cubriz Kronecker nacerá del producto. Por ejemplo, $\Delta_{jk}$ es producida cuando $\Delta$ ocupa la segunda posición. Esto tiene cierto sentido intuitivo, ya que en esta posición del producto, su subíndice $i$ se ve sustituído por el iterador $l$ del sumatorio, dejando como componentes restantes solo a $j$ y a $k$.

\newpage
