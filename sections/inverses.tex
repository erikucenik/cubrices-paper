\section{Noción de inversa.}

\subsection{Par inversa.}

Al igual que antes, buscamos inspiración en el área de las matrices, donde la matriz inversa $A^{-1}$ a otra matriz $A$ es aquélla que satisface que ${A\cdot A^{-1} = A^{-1} \cdot A = I}$. Ya ha quedado en evidencia la multiplicidad de las cubrices identidad, así que hemos de esperar lo mismo para las cubrices inversas.

Decimos que $I$ y $J$ forman un \textit{par inversa} en $AIJ$ sobre $i, j$ si $(A\cdot I \cdot J) = \delta_{ij}$. Podemos afirmar lo mismo para pares en otros órdenes del producto ($IAJ$, $IJA$) y sobre otras cubrices kronecker ($d_{jk}$, $d_{ik}$). Nótese que la conmutatividad no está asegurada.

\subsection{Todos los pares inversas.}

Tras un desarrollo algo repetitivo, recogeremos en la siguiente tabla todos los pares inversa no redundantes existentes.

\vspace{0.5cm}

\begin{tabular}{ |c|c|c|c| } 
	\hline
		  & $\delta_{ij}$      & $\delta_{jk}$       & $\delta_{ik}$ \\
	\hline
	$AIJ$ & \makecell{$I_{ijk} = \delta_{ij} A_{ijk}^{-1}$ \\ $J_{ijk} = \delta_{ij}$}                          & \makecell{$I_{ijk} = \delta_{jk}$ \\ $J_{ijk} = \delta_{jk} A_{ijk}^{-1}$}                          & \makecell{$I_{ijk} = \delta_{ik} \sum\limits^n_{l=1}\delta_{il} A_{lll}^{-1}$ \\ $J_{ijk} = \delta_{ik}$} \\
	\hline
	$IAJ$ & \makecell{$I_{ijk} = \delta_{ij} A_{ijk}^{-1}$ \\ $J_{ijk} = \delta_{ij}$}                          & \makecell{$I_{ijk} = \delta_{jk} \sum\limits^n_{l=1} \delta_{jl} A_{lll}^{-1}$ \\ $J_{ijk} = \delta_{jk}$} & \makecell{$I_{ijk} = \delta_{ik}$ \\ $J_{ijk} = \delta_{ik} A_{ijk}^{-1}$} \\
	\hline
	$IJA$ & \makecell{$I_{ijk} = \delta_{ij} \sum\limits^n_{l=1} \delta_{il} A_{lll}^{-1}$ \\ $J_{ijk} = \delta_{ij}$} & \makecell{$I_{ijk} = \delta_{jk} A_{ijk}^{-1}$ \\ $J_{ijk} = \delta_{jk}$}                          & \makecell{$I_{ijk} = \delta_{ik}$ \\ $J_{ijk} = \delta_{ik} A_{ijk}^{-1}$} \\
	\hline
\end{tabular}

\vspace{0.5cm}

Es fácil notar que con unos simples cambios de nombre podemos obtener un resultado mucho más limpio. Sea $(A^{-1})_{ijk} = (A_{ijk})^{-1}$ y $\Delta = (\delta_{ij}\delta_{jk})$. Podemos repartir sus productos con $A$ en distintas columnas según la kronecker que resulte del mismo.

\vspace{0.5cm}
\begin{tabular}{ |c|c|c| } 
	\hline
	$\delta_{ij}$ & $\delta_{jk}$ & $\delta_{ik}$ \\
	\hline
	\makecell{$(A \cdot A^{-1} \cdot \Delta)$ \\ $(A^{-1} \cdot A \cdot \Delta)$} &
	\makecell{$(A \cdot \Delta \cdot A^{-1})$ \\ $(A^{-1} \cdot \Delta \cdot A)$} &
	\makecell{$(\Delta \cdot A \cdot A^{-1})$ \\ $(\Delta \cdot A^{-1} \cdot A)$} \\
	\hline
\end{tabular}
\vspace{0.5cm}

Nótese la exclusión de la diagonal positiva de la primera tabla presentada. Aquéllas expresiones no pueden ser transformadas en términos tan limpios. Si bien podríamos afirmar que $((\delta_{ab} A^{-1}_{aaa}, (\delta_{ab}))$ constituye un par inversa para esos ordenes, consideramos que la belleza ha de primar sobre la completitud.

Analizando la última tabla, parece evidente que conmutar $A$ con $A^{-1}$ no altera el resultado. En otras palabras: es la posición de $\Delta$ la que determina qué cubriz Kronecker nacerá del producto. Por ejemplo, $\delta_{jk}$ es producida cuando $\Delta$ ocupa la segunda posición. Esto tiene cierto sentido intuitivo, ya que en esta posición del producto, su subíndice $i$ se ve sustituído por el iterador $l$ del sumatorio, dejando como componentes restantes solo a $j$ y a $k$.

\newpage
