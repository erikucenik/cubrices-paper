\section{Noción de identidad.}

La matriz identidad $I$ es aquélla que cumple que $AI = IA = A$. Es interesante buscar una noción similar en las cubrices.

\subsection{Unicidad de I.}

Sean $A, I \in M_{n} (\mathbb{K})$ tal que $IIA = A$. Partimos de tres asunciones:

\begin{itemize}
	\item Ningún elemento de $A$ es nulo.
	\item Los elementos de $I$ no dependen de $A$.
	\item La matriz identidad $I$ es única.
\end{itemize}

Por definición:

$$A_{ijk} = (I \cdot I \cdot A)_{ijk} = \sum\limits_{l=1}^{n} I_{ilk} \cdot I_{ljk} \cdot A_{ijl}$$

y para que esto sea cierto

\begin{itemize}
	\item $I_{ilk}$ debe ser $1$ para $l = j$ y $0$ para $l \neq j$. 
	\item $I_{ljk}$ debe ser $1$ para $l = i$ y $0$ para $l \neq i$.
\end{itemize}

Pero habrá casos como $I_{122}$ donde no será posible cumplir ambas condiciones a la vez. (Véase el Anexo para un desarrollo completo en $M_2 (\mathbb{K})$ si se precisa esclarecer el patrón).

Al menos una de nuestras suposiciones debe estar errada, y sería óptimo que fuese la tercera (sobre la unicidad de la identidad).

\newpage

\subsection{La identidad como par.}

Decimos que $I, J \in M_{n} (\mathbb{K})$ forman un \textit{par identidad} si cumplen que $AIJ = A$ (exploraremos más adelante la influencia del orden de los factores). Nuevamente utilizamos la definición del producto:

$$A_{ijk} = (AIJ)_{ijk} = \sum\limits_{l=1}^{n} A_{ilk} I_{ljk} J_{ijl}$$

Esta vez las condiciones que han de satisfacerse son:

\begin{itemize}
	\item $I_{ljk} = J_{ijl}^{-1}$ para $l = j$ y $I_{ljk} J_{ijl} = 0$ para $l \neq j$.
	\item $J_{ijl} = I_{ljk}^{-1}$ para $l = k$ y $J_{ijl} I_{ljk} = 0$ para $l \neq k$.
\end{itemize}

Dos cubrices cualquiera que cumplan estas condiciones serán consideradas un \textit{par identidad}, y cumplirán que $AIJ = A$. Es evidente entonces que si $\mathbb{K}$ es un cuerpo, existirán infinitos pares identidad, mientras que si es un anillo unitario conmutativo, solo existirán los que a continuación presentamos.

\subsection{Las tres Kronecker.}

Por comodidad y estandarización, resulta inmediata la idea de establecer que todos los términos que deban ser el inverso de otros términos sean iguales a $1$, mientras que todos los que deban anularse con otro sean iguales a $0$.

Es decir, que para que $(AIJ)_{ijk} = A_{ijk}$, debe cumplirse que $I_{ljk} = J_{ijl} = \delta_{jl}$, donde

\begin{equation}
	\delta_{ab} =
	\begin{cases}
		1 & \text{si } a = b \\
		0 & \text{si } a \neq b
	\end{cases}
\end{equation}

Es posible independizar estas expresiones del subíndice $l$ al notar que en $I_{ljk}$, que $l$ sea igual a $j$ es equivalente a que $i = j$ (al ser los subíndices mudos), produciendo la expresión:

$$I_{ijk} = \delta_{ij}$$

Teniendo en cuenta que el par identidad no es conmutativo, es sencillo hacer un desarrollo similar tanto para $I_2 A J_2$ como para $I_3 J_3 A$ (conmutar la posición de $I$ y $J$ no produce nuevas cubrices, solo conmuta sus nombres). Tomando $\Delta_{ab} = (\delta_{ab})$, obtenemos que: %, de lo cual obtendremos que:

$$A = A \Delta_{ij} \Delta_{jk} = \Delta_{ij} A \Delta_{ik} = \Delta_{jk} \Delta_{ik} A$$

%\begin{itemize}
%	\item $A I_1 J_1 = A \Leftrightarrow I_1 = \delta_{ij}$ y $J_1 = \delta_{jk}$.
%	\item $I_2 A J_2 = A \Leftrightarrow I_2 = \delta_{ij}$ y $J_2 = \delta_{ik}$.
%	\item $I_3 J_3 A = A \Leftrightarrow I_3 = \delta_{jk}$ y $J_3 = \delta_{ik}$.
%\end{itemize}

\newpage

Con esto concluímos que existen tres pares identidad estándar fundamentados sobre las tres Kronecker. No es inmediatamente obvio qué patrón se puede esclarecer de este resultado. Sea como fuere, es intrigante observar las cubrices dibujadas por cada Kronecker. Tomemos sus representaciones tridimensionales completas.

\begin{figure}[H]
	\includegraphics[width=\linewidth]{media/kroneckers.png}
	\caption{Representación tridimensional completa de $\delta_{ij}$, $\delta{jk}$ y $\delta{ik}$ (celdas iguales a $1$ en blanco e iguales a $0$ en negro).}
\end{figure}

Puede ser útil reagrupar y renombrar los términos de la siguiente forma. Si $I_{ijk} = 1$, entonces:

\begin{itemize}
	\item $A = A \Delta_{ij} I = A I \Delta_{jk}$.
	\item $A = \Delta_{ij} A I = I A \Delta_{ik}$.
	\item $A = \Delta_{jk} I A = I \Delta_{ik} A$.
\end{itemize}

\newpage
