\section{Introduction.} \label{intro}

The concept of matrices is, without a doubt, inseparable from linear algebra, as it constitutes a fundamental structure in physics, computer science, artificial intelligence and many more fields of knowledge.

The root of this relevance lies in how simple it makes working with large datasets when processing information, for example, via matrix multiplication.

In this brief article we will investigate a possible way of extrapolating this concept into higher dimensions. In this task we'll find ourselves needing to define a more general notion of \textit{product}, which will carry problems like the existance of \textit{non-unique identity elements} and the need to develop a theory on the meaning of \textit{inverse elements}.

In this article we'll focus on the three-dimensional particular case (whose elements we'll call ``cubrices''). Having studied cubrices' behaviour, it'll be possible to extrapolate the idea to n dimensions in the future.

This theoretical exercise doesn't aspire to find great use cases to these mathematical objects, but to explore human thought structure's borders. Despite that, we're sure that they may be convenient in some contexts.
