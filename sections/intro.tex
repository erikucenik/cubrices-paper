\section{Introducción.} \label{intro}

El concepto de matriz es, sin lugar a dudas, inseparable del álgebra lineal y constituye una estructura fundamental en la física, la computación, la inteligencia artificial e innumerables campos de conocimiento.

La raíz de esta relevancia yace en la facilidad que ofrece para trabajar con grandes conjuntos de información en el procesamiento de datos, por ejemplo, a través del producto matricial.

En este breve artículo investigaremos una posible extrapolación de este concepto a dimensiones superiores. En esta tarea nos hallaremos en la necesidad de definir una noción de \textit{producto} más general que traerá consigo problemas como la existencia de \textit{elementos neutros no únicos} y la necesidad de desarrollar una teoría sobre el significado de los \textit{elementos inversos}.

%Partiremos de una extrapolación a la tercera dimensión (a cuyos elementos bautizaremos ``cubrices''), y cuando se haya estudiado con propiedad su naturaleza, pondremos a prueba la resiliencia de nuestra creación extrapolando hacia n dimensiones.

En este artículo nos centraremos en el caso particular tridimensional (a cuyos elementos bautizaremos ``cubrices''). Estudiado el comportamiento de las cubrices, en un futuro será posible extrapolar la idea a n dimensiones.

Este desarrollo teórico no aspira a encontrar grandes usos para estos objetos matemáticos, sino más bien explorar las fronteras de la estructura del pensamiento humano. Por supuesto, estamos seguros de que a pesar de ello podrán resultar convenientes en algún contexto.
