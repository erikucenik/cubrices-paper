\section{Definiciones.}

\subsection{La estructura del $\mathbb{K}$-álgebra $M_n (\mathbb{K})$.}

Definimos $M_{m\times n\times o} (\mathbb{K}) = \{ [\mu_{ijk}] \in \mathbb{K}, 1 \le i,j,k; i \le m, j \le n, k \le o\}$ como el conjunto de las \textit{cubrices} y $M_n (\mathbb{K}) = M_{n\times n\times n} (\mathbb{K})$ como el conjunto de las \textit{cubrices cúbicas}. Por convención, una cubriz se representa con una letra griega mayúscula ($A, B, \Gamma, \Delta \in M_{m\times n\times o}$) y sus elementos se representan con letras griegas mayúsculas o minúsculas y su respectivo subíndice ($A_{ijk} = \alpha_{ijk}, B_{ijk} = \beta_{ijk}, \Gamma_{ijk} = \gamma_{ijk}, \Delta_{ijk} = \delta_{ijk}$).

%Sea $M_{m\times n\times o} (\mathbb{K}) = \{ [\mu_{ijk}] \in \mathbb{K}, 1 \le i,j,k \le m,n,o\}$ 

Podemos definir dos operaciones:

\begin{itemize}
	\item Suma de cubrices. $+: M_{m\times n\times o} \times M_{m\times n\times o} \rightarrow M_{m\times n\times o}, (A, B) \mapsto \Gamma$ tal que $\Gamma_{ijk} = A_{ijk} + B_{ijk}$.
	\item Producto por un escalar. $\cdot: \mathbb{K} \times M_{m\times n\times o} \rightarrow M_{m\times n\times o}, (a, A) \mapsto B$ tal que $B_{ijk} = a \cdot A_{ijk}$.
\end{itemize}

Si $\mathbb{K}$ es un anillo, la n-upla $M_n (\mathbb{K}) = (M_n, \mathbb{K}, +, \cdot)$ es un $\mathbb{K}$-módulo. Si $\mathbb{K}$ es además un cuerpo, $M_n (\mathbb{K})$ será un $\mathbb{K}$-espacio vectorial.

\newpage

Tenemos ahora la suficiente infraestructura para definir el \textit{producto entre cubrices}:

$$\cdot: M_{m\times n\times o} (\mathbb{K}) \times M_{p\times q\times r} (\mathbb{K}) \times M_{s\times t\times u} (\mathbb{K}) \rightarrow M_{v\times w\times x} (\mathbb{K}), (A, B, \Gamma) \mapsto \Delta$$ tal que se cumpla que:

\begin{itemize}
	\item $\Delta_{ijk} = \sum\limits_{l=1}^{n} A_{ilk} \cdot B_{ljk} \cdot \Gamma_{ijl}$.
	\item $n = p = u$.
	\item $v = \min\{m, s\}$.
	\item $w = \min\{q, t\}$.
	\item $x = \min\{o, r\}$.
\end{itemize}

Por supuesto, la n-upla $M_n (\mathbb{K}) = (M_n, \mathbb{K}, \cdot)$ es un $\mathbb{K}$-álgebra.

\subsection{Propiedades de las cubrices.}

Resulta trivial demostrar las siguientes propiedades:

\begin{itemize}
	\item Conmutatividad de la suma: $A + B = B + A$.
	\item Asociatividad de la suma: $(A + B) + \Gamma = A + (B + \Gamma)$.
	\item Asociatividad del producto por un escalar: $a\cdot (b\cdot A) = (a\cdot b)A$.
	\item Conmutatividad del producto por un escalar: $a\cdot (b\cdot A) = b\cdot (a\cdot A)$.
	\item Distributividad del escalar: $a\cdot (A + B) = a\cdot A + a\cdot B$.
	\item Distributividad de la cubriz: $(a + b)\cdot A = a\cdot A + b\cdot A$.
	\item Asociatividad del producto cubriz: $(A \cdot B \cdot \Gamma) \cdot \Delta \cdot E = A \cdot (B \cdot \Gamma \cdot \Delta) \cdot E = A \cdot B \cdot (\Gamma \cdot \Delta \cdot E)$.
	\item Existencia de una \textit{cubriz nula}: $0_{ijk} = 0$.
	\item Existencia de \textit{cubrices opuestas}: $A + (-A) = 0 \Leftrightarrow (-A)_{ijk} = -(A_{ijk}) \forall A$.
\end{itemize}

Gracias a las propiedades asociativas podemos omitir el uso de paréntesis por comodidad. Por ejemplo, $A \cdot (B \cdot \Gamma \cdot \Delta) \cdot E = A \cdot B \cdot \Gamma \cdot \Delta \cdot E$. Omitiremos también el símbolo del producto: $(a\cdot b)\cdot (A \cdot B \cdot \Gamma) = (ab)(AB\Gamma)$.

Es notable la ausencia de una noción de \textit{cubriz identidad} y otra de \textit{cubriz inversa}. Estos conceptos están sujetos a ciertas sutilezas, por lo que antes de explorarlos nos aseguraremos de haber obtenido una buena intuición sobre la apariencia de las cubrices y sus operaciones.
