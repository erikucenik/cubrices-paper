\section{Definitions.} \label{defs}

\subsection{The structure of the $M_n (\mathbb{K})$ $\mathbb{K}$-algebra.} \label{defs-structure}

We define $M_{m\times n\times o} (\mathbb{K}) = \{ [\mu_{ijk}] \in \mathbb{K}, 1 \le i \le m, 1 \le j \le n, 1 \le k \le o\}$ to be the set of \textit{cubrices} and $M_n (\mathbb{K}) = M_{n\times n\times n} (\mathbb{K})$ to be the set of \textit{cubic cubrices}. By convention, a cubrix is represented by a capital greek letter ($A, B, \Gamma, \Delta \in M_{m\times n\times o}$) and its elements are represented by either upper-case or lower-case greek letters with their respective subindices ($A_{ijk} = \alpha_{ijk}, B_{ijk} = \beta_{ijk}, \Gamma_{ijk} = \gamma_{ijk}, \Delta_{ijk} = \delta_{ijk}$).

We may define two operations:

\begin{itemize}
	\item Cubrix addition. $+: M_{m\times n\times o} \times M_{m\times n\times o} \rightarrow M_{m\times n\times o}, (A, B) \mapsto \Gamma$ tal que $\Gamma_{ijk} = A_{ijk} + B_{ijk}$.
	\item Scalar product. $\cdot: \mathbb{K} \times M_{m\times n\times o} \rightarrow M_{m\times n\times o}, (a, A) \mapsto B$ such that $B_{ijk} = a \cdot A_{ijk}$.
\end{itemize}

If $\mathbb{K}$ is a ring, the n-uple $M_n (\mathbb{K}) = (M_n, \mathbb{K}, +, \cdot)$ is a $\mathbb{K}$-module. If $\mathbb{K}$ is also a field, $M_n (\mathbb{K})$ will be a $\mathbb{K}$-vector space.

\newpage

We now have enough infrastructure to define the \textit{cubrix product}:

$$\cdot: M_{m\times n\times o} (\mathbb{K}) \times M_{p\times q\times r} (\mathbb{K}) \times M_{s\times t\times u} (\mathbb{K}) \rightarrow M_{v\times w\times x} (\mathbb{K}), (A, B, \Gamma) \mapsto \Delta$$ such that:

\begin{itemize}
	\item $\Delta_{ijk} = \sum\limits_{l=1}^{n} A_{ilk} \cdot B_{ljk} \cdot \Gamma_{ijl}$.
	\item $n = p = u$.
	\item $v = \min\{m, s\}$.
	\item $w = \min\{q, t\}$.
	\item $x = \min\{o, r\}$.
\end{itemize}

Of course, the n-uple $M_n (\mathbb{K}) = (M_n, \mathbb{K}, \cdot)$ is a $\mathbb{K}$-algebra.

\subsection{Properties of cubrices.} \label{defs-properties}

It's trivial to proove the following properties:

\begin{itemize}
	\item Addition commutativity: $A + B = B + A$.
	\item Addition associativity: $(A + B) + \Gamma = A + (B + \Gamma)$.
	\item Scalar product associativity: $a\cdot (b\cdot A) = (a\cdot b)A$.
	\item Scalar product commutativity: $a\cdot (b\cdot A) = b\cdot (a\cdot A)$.
	\item Scalar distributivity: $a\cdot (A + B) = a\cdot A + a\cdot B$.
	\item Cubrix distributivity: $(a + b)\cdot A = a\cdot A + b\cdot A$.
	\item Cubrix product associativity: $(A \cdot B \cdot \Gamma) \cdot \Delta \cdot E = A \cdot (B \cdot \Gamma \cdot \Delta) \cdot E = A \cdot B \cdot (\Gamma \cdot \Delta \cdot E)$.
	\item Existance of the \textit{null cubrix}: $0_{ijk} = 0$.
	\item Existance of \textit{opposite cubrices}: $A + (-A) = 0 \Leftrightarrow (-A)_{ijk} = -(A_{ijk}) \forall A$.
\end{itemize}

Thanks to the associative properties we are able to omit the use of parenthesis. For example, $A \cdot (B \cdot \Gamma \cdot \Delta) \cdot E = A \cdot B \cdot \Gamma \cdot \Delta \cdot E$. We may also omit the product's symbol: $(a\cdot b)\cdot (A \cdot B \cdot \Gamma) = (ab)(AB\Gamma)$.

It's worth noting that we are missing a notion for an \textit{identity cubrix} and an \textit{inverse cubrix}. These concepts are subject to some nuance, so before exploring them we shall make sure to have obtained a good intuition over the appearance of cubrices and their operations.
