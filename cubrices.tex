\documentclass[a4paper, titlepage]{article}
\title{Sobre cubrices y resultados de la extrapolación matricial.}
\author{Erik López de la Fuente}
\date{\today}

\usepackage[table]{xcolor}
\usepackage[utf8]{inputenc}
\usepackage[spanish]{babel}
\usepackage{amsmath}
\usepackage{amsfonts}
\usepackage{enumerate}
\usepackage{graphicx}
\usepackage{float}
\usepackage{makecell}
\usepackage{hyperref}
\usepackage{listings}

\counterwithin*{equation}{section}
\counterwithin*{equation}{subsection}
\counterwithin*{equation}{subsubsection}

\begin{document}

\maketitle

\begin{abstract}
	Una exploración sobre cómo extrapolar el concepto de matriz a la tercera dimensión (\textit{cubriz}), con el correspondiente análisis de sus propiedades y nociones básicas (producto, par identidad, par inverso, representación gráfica...).
\end{abstract}

\tableofcontents
\newpage

\section{Introducción.} \label{intro}

El concepto de matriz es, sin lugar a dudas, inseparable del álgebra lineal y constituye una estructura fundamental en la física, la computación, la inteligencia artificial e innumerables campos de conocimiento.

La raíz de esta relevancia yace en la facilidad que ofrece para trabajar con grandes conjuntos de información en el procesamiento de datos, por ejemplo, a través del producto matricial.

En este breve artículo investigaremos una posible extrapolación de este concepto a dimensiones superiores. En esta tarea nos hallaremos en la necesidad de definir una noción de \textit{producto} más general que traerá consigo problemas como la existencia de \textit{elementos neutros no únicos} y la necesidad de desarrollar una teoría sobre el significado de los \textit{elementos inversos}.

%Partiremos de una extrapolación a la tercera dimensión (a cuyos elementos bautizaremos ``cubrices''), y cuando se haya estudiado con propiedad su naturaleza, pondremos a prueba la resiliencia de nuestra creación extrapolando hacia n dimensiones.

En este artículo nos centraremos en el caso particular tridimensional (a cuyos elementos bautizaremos ``cubrices''). Estudiado el comportamiento de las cubrices, en un futuro será posible extrapolar la idea a n dimensiones.

Este desarrollo teórico no aspira a encontrar grandes usos para estos objetos matemáticos, sino más bien explorar las fronteras de la estructura del pensamiento humano. Por supuesto, estamos seguros de que a pesar de ello podrán resultar convenientes en algún contexto.
            % Cojonudo
\section{Definitions.} \label{defs}

\subsection{The structure of the $M_n (\mathbb{K})$ $\mathbb{K}$-algebra.} \label{defs-structure}

We define $M_{m\times n\times o} (\mathbb{K}) = \{ [\mu_{ijk}] \in \mathbb{K}, 1 \le i \le m, 1 \le j \le n, 1 \le k \le o\}$ to be the set of \textit{cubrices} and $M_n (\mathbb{K}) = M_{n\times n\times n} (\mathbb{K})$ to be the set of \textit{cubic cubrices}. By convention, a cubrix is represented by a capital greek letter ($A, B, \Gamma, \Delta \in M_{m\times n\times o}$) and its elements are represented by either upper-case or lower-case greek letters with their respective subindices ($A_{ijk} = \alpha_{ijk}, B_{ijk} = \beta_{ijk}, \Gamma_{ijk} = \gamma_{ijk}, \Delta_{ijk} = \delta_{ijk}$).

We may define two operations:

\begin{itemize}
	\item Cubrix addition. $+: M_{m\times n\times o} \times M_{m\times n\times o} \rightarrow M_{m\times n\times o}, (A, B) \mapsto \Gamma$ such that $\Gamma_{ijk} = A_{ijk} + B_{ijk}$.
	\item Scalar product. $\cdot: \mathbb{K} \times M_{m\times n\times o} \rightarrow M_{m\times n\times o}, (a, A) \mapsto B$ such that $B_{ijk} = a \cdot A_{ijk}$.
\end{itemize}

If $\mathbb{K}$ is a ring, the n-uple $M_n (\mathbb{K}) = (M_n, \mathbb{K}, +, \cdot)$ is a $\mathbb{K}$-module. If $\mathbb{K}$ is also a field, $M_n (\mathbb{K})$ will be a $\mathbb{K}$-vector space.

\newpage

We now have enough infrastructure to define the \textit{cubrix product}:

$$\cdot: M_{m\times n\times o} (\mathbb{K}) \times M_{p\times q\times r} (\mathbb{K}) \times M_{s\times t\times u} (\mathbb{K}) \rightarrow M_{v\times w\times x} (\mathbb{K}), (A, B, \Gamma) \mapsto \Delta$$ such that:

\begin{itemize}
	\item $\Delta_{ijk} = \sum\limits_{l=1}^{n} A_{ilk} \cdot B_{ljk} \cdot \Gamma_{ijl}$.
	\item $n = p = u$.
	\item $v = \min\{m, s\}$.
	\item $w = \min\{q, t\}$.
	\item $x = \min\{o, r\}$.
\end{itemize}

Of course, the n-uple $M_n (\mathbb{K}) = (M_n, \mathbb{K}, \cdot)$ is a $\mathbb{K}$-algebra.

\subsection{Properties of cubrices.} \label{defs-properties}

It's trivial to proove the following properties:

\begin{itemize}
	\item Addition commutativity: $A + B = B + A$.
	\item Addition associativity: $(A + B) + \Gamma = A + (B + \Gamma)$.
	\item Scalar product associativity: $a\cdot (b\cdot A) = (a\cdot b)A$.
	\item Scalar product commutativity: $a\cdot (b\cdot A) = b\cdot (a\cdot A)$.
	\item Scalar distributivity: $a\cdot (A + B) = a\cdot A + a\cdot B$.
	\item Cubrix distributivity: $(a + b)\cdot A = a\cdot A + b\cdot A$.
	\item Cubrix product associativity: $(A \cdot B \cdot \Gamma) \cdot \Delta \cdot E = A \cdot (B \cdot \Gamma \cdot \Delta) \cdot E = A \cdot B \cdot (\Gamma \cdot \Delta \cdot E)$.
	\item Existance of the \textit{null cubrix}: $0_{ijk} = 0$.
	\item Existance of \textit{opposite cubrices}: $A + (-A) = 0 \Leftrightarrow (-A)_{ijk} = -(A_{ijk}) \forall A$.
\end{itemize}

Thanks to the associative properties we are able to omit the use of parenthesis. For example, $A \cdot (B \cdot \Gamma \cdot \Delta) \cdot E = A \cdot B \cdot \Gamma \cdot \Delta \cdot E$. We may also omit the product's symbol: $(a\cdot b)\cdot (A \cdot B \cdot \Gamma) = (ab)(AB\Gamma)$.

It's worth noting that we are missing a notion for an \textit{identity cubrix} and an \textit{inverse cubrix}. These concepts are subject to some nuance, so before exploring them we shall make sure to have obtained a good intuition over the appearance of cubrices and their operations.
      % Cojonudo
\section{Graphical representations.}

\subsection{Two-dimensional.}

We can represent any cubrix as a series of matrices separated by vertical bars, where each matrix would be a ``slice'' ($k=1$, $k=2$...) from the cubrix. This is the standard print representation.

\[ \left(
\begin{array}{c c c | c | c c c}
	\alpha_{111} & \cdots & \alpha_{1n1} &        & \alpha_{11o} & \cdots & \alpha_{1no} \\
	\vdots       &        & \vdots       & \cdots & \vdots       &        & \vdots       \\
	\alpha_{m11} & \cdots & \alpha_{mn1} &        & \alpha_{m1o} & \cdots & \alpha_{mno} \\
\end{array} \right)
\]

\subsection{Superficial three-dimensional.}

Sometimes it will suffice with a big-picture view of the cubrix which doesn't go into detail. For this purpose, we can organize each element in voxels which make up a rectangular prism.

\begin{figure}[H]
	\includegraphics[width=\linewidth]{media/tridimensional_sup.png}
	\caption{Superficial three-dimensional representation of a cubrix with elements $\alpha_{ijk}$.}
\end{figure}

\newpage

\subsection{Complete three-dimensional.}

For a complete view which preserves the three-dimensional structure, we can separate each ``slice'' of the cubrix along a certain subindex. This representation can aid in pattern recognition.

\begin{figure}[H]
	\includegraphics[width=\linewidth]{media/tridimensional_comp.png}
	\caption{Complete three-dimensional representation of a cubrix with elements $\alpha_{ijk}$.}
\end{figure}

\subsection{Product.}

Under the definition given, element $ijk$ of cubrix $\Delta = (AB\Gamma)$ will be equal to the product of $A$'s $i$th row by $B$'s $j$th column by $\Gamma$'s $k$th ``depth''.

\begin{figure}[H]
	\includegraphics[width=\linewidth]{media/product.png}
	\caption{Superficial three-dimensional representation of the product $\Delta = (AB\Gamma)$.}
\end{figure}

This explains the restrictions marked by the cubrices' dimensions. In the sum's iterative process, it's necessary for the number of columns of $A$ ($n$) to equal the number of rows of $B$ ($p$) and to equal the number of ``depths'' of $\Gamma$ ($u$). Furthermore, $\Delta$ may not have more rows than $A$ nor than $\Gamma$ (otherwise, we would access non-existant values), nor more columns than $B$ or $\Gamma$, nor more ``depths'' than $A$ or $B$.
  % Cojonudo
\section{Noción de identidad.}

La matriz identidad $I$ es aquélla que cumple que $AI = IA = A$. Es interesante buscar una noción similar en las cubrices.

\subsection{Unicidad de I.}

Sean $A, I \in M_{n} (\mathbb{K})$ tal que $IIA = A$. Partimos de tres asunciones:

\begin{itemize}
	\item Ningún elemento de $A$ es nulo.
	\item Los elementos de $I$ no dependen de $A$.
	\item La matriz identidad $I$ es única.
\end{itemize}

Por definición:

$$A_{ijk} = (I \cdot I \cdot A)_{ijk} = \sum\limits_{l=1}^{n} I_{ilk} \cdot I_{ljk} \cdot A_{ijl}$$

y para que esto sea cierto

\begin{itemize}
	\item $I_{ilk}$ debe ser $1$ para $l = j$ y $0$ para $l \neq j$. 
	\item $I_{ljk}$ debe ser $1$ para $l = i$ y $0$ para $l \neq i$.
\end{itemize}

Pero habrá casos como $I_{122}$ donde no será posible cumplir ambas condiciones a la vez. (Véase el Anexo para un desarrollo completo en $M_2 (\mathbb{K})$ si se precisa esclarecer el patrón).

Al menos una de nuestras suposiciones debe estar errada, y sería óptimo que fuese la tercera (sobre la unicidad de la identidad).

\newpage

\subsection{La identidad como par.}

Decimos que $I, J \in M_{n} (\mathbb{K})$ forman un \textit{par identidad} si cumplen que $AIJ = A$ (exploraremos más adelante la influencia del orden de los factores). Nuevamente utilizamos la definición del producto:

$$A_{ijk} = (AIJ)_{ijk} = \sum\limits_{l=1}^{n} A_{ilk} I_{ljk} J_{ijl}$$

Esta vez las condiciones que han de satisfacerse son:

\begin{itemize}
	\item $I_{ljk} = J_{ijl}^{-1}$ para $l = j$ y $I_{ljk} J_{ijl} = 0$ para $l \neq j$.
	\item $J_{ijl} = I_{ljk}^{-1}$ para $l = k$ y $J_{ijl} I_{ljk} = 0$ para $l \neq k$.
\end{itemize}

Dos cubrices cualquiera que cumplan estas condiciones serán consideradas un \textit{par identidad}, y cumplirán que $AIJ = A$. Es evidente entonces que si $\mathbb{K}$ es un cuerpo, existirán infinitos pares identidad, mientras que si es un anillo unitario conmutativo, solo existirán los que a continuación presentamos.

\subsection{Las tres Kronecker.}

Por comodidad y estandarización, resulta inmediata la idea de establecer que todos los términos que deban ser el inverso de otros términos sean iguales a $1$, mientras que todos los que deban anularse con otro sean iguales a $0$.

Es decir, que para que $(AIJ)_{ijk} = A_{ijk}$, debe cumplirse que $I_{ljk} = J_{ijl} = \delta_{jl}$, donde

\begin{equation}
	\delta_{ab} =
	\begin{cases}
		1 & \text{si } a = b \\
		0 & \text{si } a \neq b
	\end{cases}
\end{equation}

Es posible independizar estas expresiones del subíndice $l$ al notar que en $I_{ljk}$, que $l$ sea igual a $j$ es equivalente a que $i = j$ (al ser los subíndices mudos), produciendo la expresión:

$$I_{ijk} = \delta_{ij}$$

Teniendo en cuenta que el par identidad no es conmutativo, es sencillo hacer un desarrollo similar tanto para $I_2 A J_2$ como para $I_3 J_3 A$ (conmutar la posición de $I$ y $J$ no produce nuevas cubrices, solo conmuta sus nombres). Tomando $\Delta_{ab} = (\delta_{ab})$, obtenemos que: %, de lo cual obtendremos que:

$$A = A \Delta_{ij} \Delta_{jk} = \Delta_{ij} A \Delta_{ik} = \Delta_{jk} \Delta_{ik} A$$

%\begin{itemize}
%	\item $A I_1 J_1 = A \Leftrightarrow I_1 = \delta_{ij}$ y $J_1 = \delta_{jk}$.
%	\item $I_2 A J_2 = A \Leftrightarrow I_2 = \delta_{ij}$ y $J_2 = \delta_{ik}$.
%	\item $I_3 J_3 A = A \Leftrightarrow I_3 = \delta_{jk}$ y $J_3 = \delta_{ik}$.
%\end{itemize}

\newpage

Con esto concluímos que existen tres pares identidad estándar fundamentados sobre las tres Kronecker. No es inmediatamente obvio qué patrón se puede esclarecer de este resultado. Sea como fuere, es intrigante observar las cubrices dibujadas por cada Kronecker. Tomemos sus representaciones tridimensionales completas.

\begin{figure}[H]
	\includegraphics[width=\linewidth]{media/kroneckers.png}
	\caption{Representación tridimensional completa de $\delta_{ij}$, $\delta{jk}$ y $\delta{ik}$ (celdas iguales a $1$ en blanco e iguales a $0$ en negro).}
\end{figure}

Puede ser útil reagrupar y renombrar los términos de la siguiente forma. Si $I_{ijk} = 1$, entonces:

\begin{itemize}
	\item $A = A \Delta_{ij} I = A I \Delta_{jk}$.
	\item $A = \Delta_{ij} A I = I A \Delta_{ik}$.
	\item $A = \Delta_{jk} I A = I \Delta_{ik} A$.
\end{itemize}

\newpage
       % Otra expresión de la identidad.
\section{Notion of inverse.} \label{inverse}

\subsection{Inverse pair.} \label{inverse-pair}

As we did before, we'll find inspiration in the field of matrices, where the inverse matrix $A^{-1}$ of a nother matrix $A$ is that which satisfies that ${A\cdot A^{-1} = A^{-1} \cdot A = I}$. The multiplicity of identity cubrices has already become evident, so we should expect the same for inverse cubrices.

We say that $I$ and $J$ form an \textit{inverse pair} in $AIJ$ over $i, j$ if $(AIJ) = \Delta_{ij}$. We can say the same for pairs in other orders ($IAJ$, $IJA$) and over other kronecker cubrices ($\Delta_{jk}$, $\Delta_{ik}$). Note that commutativity isn't insured.

\subsection{All inverse pairs.} \label{inverse-all-pairs}

After a somewhat repetitive derivation (see Appendix \ref{appendix-3}), we can gather all non-redundant inverse pairs in the following table.

\bgroup
\begin{table}[H] 
\centering
\captionsetup{labelformat=empty}
\caption{(Table 1) Inverse pairs.}
\label{tabla-explicita}
\def\arraystretch{1.5}%  1 is the default, change whatever you need
\begin{tabular}{|c|c|c|c|} 
	\hline
	& $\Delta_{ij}$ & $\Delta_{jk}$ & $\Delta_{ik}$ \\
	\hline

	$AIJ$
	& \begin{tabular}{c} $I_{ijk} = \delta_{ij} A_{ijk}^{-1}$ \\ $J_{ijk} = \delta_{ij}$ \end{tabular}
	& \begin{tabular}{c} $I_{ijk} = \delta_{jk}$ \\ $J_{ijk} = \delta_{jk} A_{ijk}^{-1}$ \end{tabular}
	& \begin{tabular}{c} $I_{ijk} = \delta_{ik} A_{kkk}^{-1}$ \\ $J_{ijk} = \delta_{ik}$ \end{tabular} \\
	\hline

	$IAJ$
	& \begin{tabular}{c} $I_{ijk} = \delta_{ij} A_{ijk}^{-1}$ \\ $J_{ijk} = \delta_{ij}$ \end{tabular}
	& \begin{tabular}{c} $I_{ijk} = \delta_{jk} A_{jjj}^{-1}$ \\ $J_{ijk} = \delta_{jk}$ \end{tabular}
	& \begin{tabular}{c} $I_{ijk} = \delta_{ik}$ \\ $J_{ijk} = \delta_{ik} A_{ijk}^{-1}$ \end{tabular} \\
	\hline

	$IJA$
	& \begin{tabular}{c} $I_{ijk} = \delta_{ij} A_{iii}^{-1}$ \\ $J_{ijk} = \delta_{ij}$ \end{tabular}
	& \begin{tabular}{c} $I_{ijk} = \delta_{jk} A_{ijk}^{-1}$ \\ $J_{ijk} = \delta_{jk}$ \end{tabular}
	& \begin{tabular}{c} $I_{ijk} = \delta_{ik}$ \\ $J_{ijk} = \delta_{ik} A_{ijk}^{-1}$ \end{tabular} \\
	\hline
\end{tabular}
\end{table}
\egroup

It's easy to note that with some simple renamings we can obtain a much cleaner result (see Appendix \ref{appendix-4}). Let $(A^{-1})_{ijk} = (A_{ijk})^{-1}$ and $\Delta = (\delta_{ij}\delta_{jk})$. We can distribute their products with $A$ in different columns according to the kronecker that comes out of each one.

\bgroup
\begin{table}[H]
\centering
\captionsetup{labelformat=empty}
\caption{(Table 2) Simplified inverse pairs.}
\label{tabla-simplificada}
\def\arraystretch{1.5}%  1 is the default, change whatever you need
\begin{tabular}{|c|c|c|c|} 
	\hline
	$\Delta_{ij}$ & $\Delta_{jk}$ & $\Delta_{ik}$ \\
	\hline
	\begin{tabular}{c} $(A \cdot A^{-1} \cdot \Delta)$ \\ $(A^{-1} \cdot A \cdot \Delta)$ \end{tabular} &
	\begin{tabular}{c} $(A \cdot \Delta \cdot A^{-1})$ \\ $(A^{-1} \cdot \Delta \cdot A)$ \end{tabular} &
	\begin{tabular}{c} $(\Delta \cdot A \cdot A^{-1})$ \\ $(\Delta \cdot A^{-1} \cdot A)$ \end{tabular} \\
	\hline
\end{tabular}
\end{table}
\egroup

Note the exclusion of the first table's positive diagonal. Those expressions can't be transformed in such clean terms. We could say that $((\delta_{ab} A^{-1}_{aaa}, (\delta_{ab}))$ forms an inverse pair for those arrangements, but we consider beauty to be more important than completeness.

Analyzing the last table, it seems evident that commuting $A$ with $A^{-1}$ doesn't alter the result. In other words: it's $\Delta$'s position the one that determines which kronecker cubrix the product will yield. For example, $\Delta_{jk}$ is produced when $\Delta$ is in the second position. This holds some intuitive sense, as in this position, its subindex $i$ gets replaced by the iterator $l$ from the sum, leaving $j$ and $k$ as the only components.
         % Reescribir un poco
\section{Programatic implementation.}

The study of this mathematical concept becomes tedious without computational assistance, so it'll be necessary to develop the tools that make it possible.

We offer the following GitHub repository created along this document: \href{https://github.com/erikucenik/cubrices}{https://github.com/erikucenik/cubrices}.

In addition, we include another repository which contains the most up-to-date version of this article: \href{https://github.com/erikucenik/cubrices-paper}{https://github.com/erikucenik/cubrices-paper}.
   % Cojonudo
\section{Conclusión}

Tras este análisis descubrimos que las cubrices conforman un conjunto verdaderamente interesante a la par que misterioso. Contrario a lo que se podría presuponer, este álgebra no es retrocompatible con el de las matrices bidimensionales, si bien es cierto que los conjuntos subyacentes sí lo son.

Hemos excluido de este artículo numerosos asuntos de vital importancia que podrían ser de interés para futuras investigaciones. Nociones como el ``determinante'', el ``rango'', o una definición de igualdad entre cubrices que hiciese quizás viable su retrocompatibilidad con las matrices.

Esperamos que el presente documento sirva como incentivo para implantar la curiosidad por este álgebra en la mente del lector.
       % Cojonudo

\newpage

\counterwithin*{equation}{section}
\counterwithin*{equation}{subsection}
\counterwithin*{equation}{subsubsection}

\appendix
\section{Appendix} \label{appendix}

\subsection{Full derivation of a hypothetical unique identity cubrix in $M_2 (\mathbb{K})$.} \label{appendix-1}

The expression $A = IIA$ can be expanded using the definition of the cubrix product:

\begin{equation}
A_{111} = I_{111} I_{111} A_{111} + I_{121} I_{211} A_{112}
\end{equation}

\begin{equation}
A_{112} = I_{112} I_{112} A_{111} + I_{122} I_{212} A_{112}
\end{equation}

\begin{equation}
A_{121} = I_{111} I_{121} A_{121} + I_{121} I_{221} A_{122}
\end{equation}

\begin{equation}
A_{122} = I_{112} I_{122} A_{121} + I_{122} I_{222} A_{122}
\end{equation}

\begin{equation}
A_{211} = I_{211} I_{111} A_{211} + I_{221} I_{211} A_{212}
\end{equation}

\begin{equation}
A_{212} = I_{212} I_{112} A_{211} + I_{222} I_{212} A_{212}
\end{equation}

\begin{equation}
A_{221} = I_{211} I_{121} A_{221} + I_{221} I_{221} A_{222}
\end{equation}

\begin{equation}
A_{222} = I_{212} I_{122} A_{221} + I_{222} I_{222} A_{222}
\end{equation}

Using this system, we get the following:

\begin{enumerate}[(a)]
	\item By $(1) \Rightarrow I_{111} = 1$.
	\item By $(8) \Rightarrow I_{222} = 1$.
	\item By $(3)$ and $(a) \Rightarrow I_{121} = 1$.
	\item By $(3)$ and $(c) \Rightarrow I_{221} = 0$.
	\item By $(6)$ and $(b) \Rightarrow I_{212} = 1$.
	\item By $(2)$ and $(e) \Rightarrow I_{122} = 1$.
\end{enumerate}

By $(8)$, either $I_{212}$ or $I_{122}$ must be equal to $0$, but by $(e)$ and $(f)$, $I_{212} = I_{122} = 1 \neq 0$. The same contradiction occurs when trying with $IAI$ and $AII$.

\subsection{Full derivation of the product by an identity pair in $M_2 (\mathbb{K})$.} \label{appendix-2}

\begin{equation}
A_{111} = A_{111} I_{111} J_{111} + A_{121} I_{211} J_{112}
\end{equation}

\begin{equation}
A_{112} = A_{112} I_{112} J_{111} + A_{122} I_{212} J_{112}
\end{equation}

\begin{equation}
A_{121} = A_{111} I_{121} J_{121} + A_{121} I_{221} J_{122}
\end{equation}

\begin{equation}
A_{122} = A_{112} I_{122} J_{121} + A_{122} I_{222} J_{122}
\end{equation}

\begin{equation}
A_{211} = A_{211} I_{111} J_{211} + A_{221} I_{211} J_{212}
\end{equation}

\begin{equation}
A_{212} = A_{212} I_{112} J_{211} + A_{222} I_{212} J_{212}
\end{equation}

\begin{equation}
A_{221} = A_{211} I_{121} J_{221} + A_{221} I_{221} J_{222}
\end{equation}

\begin{equation}
A_{222} = A_{212} I_{122} J_{221} + A_{222} I_{222} J_{222}
\end{equation}

In an analogous way, we get that:

\begin{enumerate}[(a)]
	\item By $(1) \Rightarrow I_{111} \neq 0 \neq J_{111}$ y $I_{111} = J_{111}^{-1}$.
	\item By $(8) \Rightarrow I_{222} \neq 0 \neq J_{222}$ y $I_{222} = J_{222}^{-1}$.
	\item By $(2)$ and $(a) \Rightarrow I_{112} = J_{111}^{-1} = I_{111} \neq 0$.
	\item By $(6)$ and $(c) \Rightarrow I_{112} = J_{211}^{-1} \neq 0 \Rightarrow J_{211} \neq 0$.
	\item By $(5)$ and $(d) \Rightarrow J_{211} = I_{111}^{-1} = J_{111}$.
	\item By $(7)$ and $(b) \Rightarrow I_{221} = J_{222}^{-1} \neq 0$.
	\item By $(3)$, $(4)$ and $(f) \Rightarrow J_{122} = I_{221}^{-1} = I_{222}^{-1} = J_{222}^{-1}$.
	\item By $(g)$ and $(b) \Rightarrow I_{222} = J_{222}^{-1} = I_{222}^{-1} \Rightarrow I_{222}^2 = 1 \Rightarrow I_{222} = 1 = J_{222}$.
\end{enumerate}

\newpage

To sum up:

\begin{itemize}
	\item $I_{221} = J_{122} = I_{222} = J_{222} = 1$.
	\item $J_{211}^{-1} = I_{111} = J_{111}^{-1} = I_{112}$.
	\item The rest of the terms:

	\begin{itemize}
		\item $I_{211} J_{112} = 0$.
		\item $I_{212} J_{112} = 0$.
		\item $I_{121} J_{121} = 0$.
		\item $I_{122} J_{121} = 0$.
		\item $I_{211} J_{212} = 0$.
		\item $I_{212} J_{212} = 0$.
		\item $I_{121} J_{221} = 0$.
		\item $I_{122} J_{221} = 0$.
	\end{itemize}
\end{itemize}

To standardize, we can assume that the elements that don't cancel with others will be equal to $1$, whereas the ones which do, will be equal to $0$.

\begin{itemize}
	\item $1 = I_{221} = J_{122} = I_{222} = J_{222} = J_{211}^{-1} = I_{111} = J_{111}^{-1} = I_{112}$.
	\item $0 = I_{211} = J_{112} = I_{212} = I_{121} = J_{121} = I_{122} = J_{212} = J_{221}$.
\end{itemize}

We see that the following is always satisfied:

\begin{itemize}
	\item If $j = 1 \rightarrow I_{1jk} = J_{ij1} = 1$ and $I_{2jk} = J_{ij2} = 0$.
	\item If $j = 2 \rightarrow I_{1jk} = J_{ij1} = 0$ and $I_{2jk} = J_{ij2} = 1$.
\end{itemize}

The pattern becomes trivial if we see what happens with a $3 \times 3 \times 3$ cubrix:

$$(AIJ)_{ijk} = A_{i1k} I_{1jk} J_{ij1} + A_{i2k} I_{2jk} J_{ij2} + A_{i3k} I_{3jk} J_{ij3}$$

\begin{itemize}
	\item If $j = 1 \rightarrow I_{1jk} = J_{ij1} = 1$ and $I_{2jk} = J_{ij2} = 0$ and $I_{3jk} = J_{ij3} = 0$.
	\item If $j = 2 \rightarrow I_{1jk} = J_{ij1} = 0$ and $I_{2jk} = J_{ij2} = 1$ and $I_{3jk} = J_{ij3} = 0$.
	\item If $j = 3 \rightarrow I_{1jk} = J_{ij1} = 0$ and $I_{2jk} = J_{ij2} = 0$ and $I_{3jk} = J_{ij3} = 1$.
\end{itemize}

That is, for $(AIJ)_{ijk} = A_{ijk}$ to be true, then $I_{ljk} = J_{ijl} = \delta_{jl}$ must also be true.

As we discovered in section \ref{identity-pair}, we can replace the subindices and arrive to the three kroneckers.

\subsection{Full derivation of inverse pairs.} \label{appendix-3}

Given $A$, our objective is to find its inverse pairs over two components in any arrangement of factors. We will treat products which commute $I$ and $J$ as equal cases (justifying $AIJ$ will also justify $AJI$), as this will only commute the names we give to the values.

\subsubsection{$AIJ$ over $i, j$.}

By definition:

$$(AIJ)_{ijk} = \delta_{ij} = \sum\limits^{n}_{l = 1} A_{ilk} I_{ljk} J_{ijl}$$

We can explain this expression using words: ``If $i \neq j$, then $I_{ljk} J_{ijl} = 0$. If $i = j$, and if $l$ takes the value of $j$, then $I_{ljk} = A^{-1}_{ilk}$ and $J_{ijl} = 1$''. It's vital noting that the fact that $l$ takes the value of $j$ implies that the first subindex of $I_{ljk}$ also takes the value of $j$. This is equivalent to $i$ being equal to $j$ (but \textbf{only} in $I$'s case, as in other terms the iterator $l$ will occupy another subindex's position). With this in mind, expressing the same using kronecker deltas is automatic:

$$I_{ijk} = \delta_{ij} \delta_{ij} A_{ijk}^{-1} = \delta_{ij} A_{ijk}^{-1}$$
$$J_{ijk} = \delta_{ij} \approx \delta_{ij} \delta_{jk}$$

In $I$, the first $\delta_{ij}$ represents the condition imposed by the kronecker delta we aspire to get, whereas the second $\delta_{ij}$ represents the condition imposed by the iterator $l$, but expressed in terms of $I$'s subindices. Obviously, $\delta_{ij} \delta_{ij} = \delta_{ij}$.

In $J$, the term $\delta_{jk}$ represents the condition that ``...if $i = j$ but $l$ doesn't take the value of $j$...'' (which in $J$ is equivalent to saying that $k$ doesn't take the value of $j$). This term is unnecesary, as it's only required that the product gets cancelled (and not necessarily both factors), and this is already satisfied thanks to the $\delta_{ij}$ present in $I$. The imposed conditions over $I$'s and $J$'s values are perfectly represented in the following algorithm:

\begin{lstlisting}[escapeinside={(*}{*)}]
If i == j:
  If l == j:
    (*$I_{ljk} = A_{ilk}^{-1}$*)
    (*$J_{ijl} = 1$*)
  Else:
    (*$I_{ljk} J_{ijl} = 0$*)
Else:
  (*$I_{ljk} J_{ijl} = 0$*)	
\end{lstlisting}

In this article we omit the inclusion of the second factor, both for aesthetical and for practical reasons (see \ref{appendix-4}).

\subsubsection{$AIJ$ over $j, k$.}

The process is almost identical to the previous one, but commuting $I$ and $J$ (the inverse values will be contained in $J$, and the neutral ones in $I$). The reason to do this lies in the need to coordinate $I$'s and $J$'s subindices with the iterator such that we get easily malleable expressions like $\delta_{jk} \delta_{jk}$

\subsubsection{$AIJ$ over $i, k$.}

This case requires some attention. Let $a = i = k$ (if $i \neq k$, the product $I_{ijk} J_{ijk}$ will equal zero):

$$\delta_{aa} = 1 = \sum\limits_{l = 1}^{n} A_{ala} I_{lja} J_{ajl}$$

Until now, $I_{ijk}$ had always been dependant on the inverse of $A_{ijk}$, however, this case breaks that relationship. Because of that, we must find other criterion for $I$ and $J$ not to cancel. Once again, by a mixture of aesthetical reasons and process of elimination, we establish the following: when $I$'s and $J$'s indices are equal, their product with $A_{ijk}$ will yield $1$ (and when not, $0$).

It's easy to find out that $I$'s and $J$'s subindices will be equal when $l$ takes the value of $a$. So:

$$I_{aja} = A_{aaa}^{-1}$$
$$J_{aja} = 1$$

Using $i$, $j$ and $k$:

$$I_{ijk} = \delta_{ik} A_{iii}^{-1} = \delta_{ik} A_{kkk}^{-1}$$
$$J_{ijk} = \delta_{ik}$$

\subsubsection{Et cetera.}

We have been working with $AIJ$, but we could have done the same procedures over $IAJ$ and $IJA$. Every case can be solved in an analogous way to one of the three previous methods, and the final result is Table \ref{tabla-explicita} on section \ref{inverse-all-pairs}.

\subsection{Simplification of inverse pairs.} \label{appendix-4}

The expressions for inverse pairs can be simplified quite remarkably by just reordering some terms.

As our first and only example, we'll use the case of $IJA$ over $i, k$. According to Table \ref{table-explicita}:

$$I_{ijk} = \delta_{ik}$$
$$J_{ijk} = \delta_{ik} A_{ijk}^{-1}$$

Plugging that into the product's definition:

$$(IJA)_{ijk} = (\delta_{ik})_{ilk} \cdot (\delta_{ik} A_{ijk}^{-1})_{ljk} \cdot A$$

And plugging in the subindices:

$$(IJA)_{ijk} = (\delta_{ik}) \cdot (\delta_{lk} A_{ljk}^{-1}) \cdot A$$

Because $\mathbb{K}$ is associative and commutative, we can rewrite this expression such that:

$$(IJA)_{ijk} = (\delta_{ik} \delta_{lk}) \cdot A_{ljk}^{-1} \cdot A$$

And this is equivalent to having another inverse pair $(I', J')$ where:

$$I'_{ijk} = \delta_{ik} \delta_{jk}$$
$$J'_{ijk} = (A_{ijk})^{-1}$$

($l$ will be plugged into $j$ in $I$ and into $i$ in $J$).

To provide some semantic meaning to this inverse pair, we can rename $I'$ to be $\Delta$ and $J'$ to $A^{-1}$.

The same procedure is applicable to the other cases (excluding the ones on Table \ref{tabla-explicita}'s positive diagonal), producing Table \ref{tabla-simplificada} from section \ref{inverse-all-pairs}.


\end{document}
